\documentclass{article}
\usepackage[utf8]{inputenc}
\usepackage[spanish]{babel}
\usepackage{hyperref}
\usepackage{titlesec}
\usepackage{graphics}
\usepackage{apacite}
\usepackage{amsmath}
\usepackage{amsfonts}
\usepackage{amssymb}
\usepackage{graphicx}
\usepackage{xcolor}
\usepackage{titlesec}
\definecolor{color_uni}{HTML}{800404}
\bibliographystyle{apacite}
\usepackage{parskip}
\usepackage{setspace}
\usepackage{siunitx}
\usepackage{hyperref}
\usepackage[left=2.54cm, right=2.54cm, top=2.54cm, bottom=2.54cm]{geometry}


\title{Compendio de ejercicios resueltos}
\author{Francis Joao Huaman}
\date{December 2023}


\begin{document}

%--------------Inicio de documento--------------%
\begin{titlepage}
    \begin{center}
        {\LARGE \textbf{UNIVERSIDAD NACIONAL DE INGENIER\'IA}}\\
        \vspace{4mm}
        {\Large \text{FACULTAD DE INGENIER\'IA INDUSTRIAL Y DE SISTEMAS}}\\
        \vspace{4mm}

        \begin{figure}[h]
            \centering
            \includegraphics[height=8.5cm]{assets/images/Logo UNI.png}     
        \end{figure}
        \textcolor{color_uni}{\rule{\linewidth}{0.40mm}}\\
        \begin{spacing}{1}
            \vspace{0.34cm}
            {\Large \textbf{``Solucionario de compendio de ejercicios N° 2''}}\\
        \end{spacing}
        \textcolor{color_uni}{\rule{\linewidth}{0.40mm}}\\
        \vspace{0.4cm}

    \end{center}

\begin{center}
    \vspace{0.5cm}
    {\large \textbf{CURSO:} \text{Qu\'imica I} \hspace{0.5cm} \textbf{SECCIÓN:} \text{B}}\\
    \vspace{1cm}
    {\large \textbf{DOCENTE:}}\\
    \vspace{0.5cm}
    {\large \text{FUKUDA KAGAMI, Nancy Elena}}\\
    \vspace{0.8cm}
    {\large \textbf{ALUMNO:}}\\
    \vspace{0.5cm}
    {\large \text{CRUZ HUAMAN, Francis Joao}}\\
    \vspace{2cm}
    {\large \textbf{LIMA - PER\'U}}\\
    \vspace{0.4cm}
    {\large \text{2024}}\\
\end{center}    

\end{titlepage}


%-------------Contenido del documento---------------%
\begin{center}
    {\large \textbf{Solucionario de ejercicios propuestos}}
\end{center}


\begin{enumerate}
    %- 1 -%
    \item 
    \textbf{Solución:}
    \begin{enumerate}
        \item 
        Sea $n_1 = 2$ y $n_2 = 3 \rightarrow \infty$ pertenecientes al espectro visible, entonces:
        \begin{equation*}
        \begin{split}
            \frac{1}{486 \times 10^{-9}m } &= 1.097 \times 10^{7}m^{-1} \left(\frac{1}{2^{2}} - \frac{1}{n_2^{2}}\right) \\
                n_2 &= 4.0021\\
                n_2 &\thickapprox 4 \\
        \end{split}
        \end{equation*}

        \item 
        Si $\lambda = 570nm$ en el espectro visible, entonces:
        \begin{equation*}
            \begin{split}
                \frac{1}{570 \times 10^{-9}m } &= 1.097 \times 10^{7}m^{-1} \left(\frac{1}{2^{2}} - \frac{1}{n_x^{2}}\right) \\
                n_x &= 3.3320\\
                n_x &\thickapprox 3 \\
            \end{split}
        \end{equation*}
        $\therefore$ Podemos afirmar que hay transici\'on electr\'onica para $\lambda = 570nm$ , ya que $n: 2 \rightarrow 3$
    \end{enumerate}


    %- 2 -%
    \item 
    \textbf{Solución:}
    \begin{enumerate}
        \item 
        Sea un electr\'on $e$ con masa $m_e = 9.1 \times 10^{-31}Kg$ con una velocidad $v = 7000Km/s$ 
        \begin{equation*}
            \begin{split}
                \Delta x (9.1 \times 10^{-28}g) (210m/s) &\geqslant \frac{6.626 \times 10^{-34}J.s}{4\pi}\\
                \Delta x &\geqslant 2.76 \times 10^{-10}m
            \end{split}
        \end{equation*}    

        \item 
        Sea un proyectil $p$ con masa $m_p = 50g$ con una velocidad $v = 300m/s$ 
        \begin{equation*}
            \begin{split}
                \Delta x (50g) (0.009m/s) &\geqslant \frac{6.626 \times 10^{-34}J.s}{4\pi}\\
                \Delta x &\geqslant 1.17 \times 10^{-34}m
        \end{split}
        \end{equation*}
    \end{enumerate}
    
    
    %- 3 -%
    \item 
    \textbf{Solución:}
    Sea el ion ${C_{(g)}}^{5+}$ con $R_{{C_{(g)}}^{5+}} = 1.09733 \times 10^{7}m^{-1}$ y $z = 12$
    \begin{enumerate}
        \item 
        Para la longitud de onda de la cuarta linea de Brakett, entonces: $n_1 = 4$ y $n_2 = 8$
        \begin{equation*}
            \begin{split}
                \frac{1}{\lambda} &= (1.09733 \times 10^{7}m^{-1})(12^{2}) \left(\frac{1}{4^{2}} - \frac{1}{8^{2}} \right)\\
                \lambda &= 1.35 \times 10^{-8}m
            \end{split}
        \end{equation*}

        \item
        La longitud de onda del quinto nivel de energ\'ia en el espectro de Brakett, entonces: $n_1 = 4$ y $n_2 = 5$
        \begin{equation*}
            \begin{split}
                \frac{1}{\lambda} &= (1.09733 \times 10^{7}m^{-1})(12^{2}) \left(\frac{1}{4^{2}} - \frac{1}{5^{2}} \right)\\
                \lambda &= 2.81 \times 10^{-8}m
        \end{split}
        \end{equation*}
    \end{enumerate}


    %- 4 -%
    \item 
    \textbf{Solución:}
    Sea una onda que incide sobre una superficie de $Na (\lambda = 4500 \si{\angstrom} = 4500 \times 10^{-10})$ y la \\
    $E_{k_{max}} = 3.36 \times 10^{-12}erg = 3.36 \times 10^{-19}\si\joule$   %----\r{A}----%  %----\setcounter{enumii}{1}---%\\
    $$ E_{inc} = \phi + E_k$$
    \begin{enumerate}
        \item 
        \begin{equation*}
            \begin{split}
                \phi &= \frac{(6.626 \times 10^{-34}J.s)(3 \times 10^{8}m/s)}{4500 \times 10^{-10}m} - 3.36 \times 10^{-19}\si\joule\\
                \phi &= 1.057 \times 10^{-19}\si\joule
            \end{split}
        \end{equation*}

        \item
        \begin{equation*}
            \begin{split}
                \phi &= h\upsilon_{umbral}\\
                1.057 \times 10^{-19}\si\joule &= (6.626 \times 10^{-34}J.s)\upsilon_{umbral}\\
                \upsilon_{umbral} &= 1.60 \times 10^{14}s^{-1}
            \end{split}
        \end{equation*}

        \item
        \begin{equation*}
            \begin{split}
                \lambda_{m\acute{a}x} &= \frac{3 \times 10^{8}m/s}{1.6 \times 10^{14}s^{-1}}\\
                \lambda_{m\acute{a}x} &= 1.88 \times 10^{-6}m 
            \end{split}
        \end{equation*}
    \end{enumerate}


    %- 5 -%
    \item 
    \textbf{Solución:}
    \begin{enumerate}
        \item 
        \begin{equation*}
            \begin{split}
                E_{fotones} &= 13.527 eV\\
                \lambda  &= \frac{(6.626 \times 10^{-34}J.s)(3 \times 10^{8}m/s)}{2.16 \times 10^{-18}\si\joule}\\
                \lambda  &= 9.20 \times 10^{-8}m
            \end{split}
        \end{equation*}
        Calculando la constante $R_H$ para el \'atomo de hidr\'ogeno: 
        \begin{equation*}
            \begin{split}
                \frac{1}{\lambda} &= R_H \left( \frac{1}{n^{2}}\right)\\
                \frac{1}{9.20 \times 10^{-8}m} &= R_H \left( \frac{1}{1^{2}}\right)\\
                R_H &= 1.087 \times 10^{-7} m^{-1}
            \end{split}
        \end{equation*}
    \end{enumerate}
    
    
    %- 6 -%
    \item 
    \textbf{Solución:}
    Sea una estaci\'on de radio que emite $\lambda_{emisi\acute{o}n} = 25m$
    \begin{enumerate}
        \item 
        \begin{equation*}
            \begin{split}
                \upsilon &= \frac{3 \times 10^{8}m/s}{25m}\\
                \upsilon &= 1.2 \times 10^{7}s^{-1}
            \end{split}
        \end{equation*}

        \item
        \begin{equation*}
            \begin{split}
                E_{fotones} &= (6.626 \times 10^{-34}J.s)(1.2 \times 10^{7}s^{-1})\\
                E_{fotones} &= 7.95 \times 10^{-27}J
            \end{split}
        \end{equation*}

        \item
        Fotones emitidos por hora con potencia de $6KW = 6 \times 10^{3} J/s$
        \begin{equation*}
            \begin{split}
                n(fotones) &= \frac{3600s ( 6 \times 10^{3} J/s)}{7.95 \times 10^{-27}J}\\
                n(fotones) &= 2.72 \times 10^{33}
            \end{split}
        \end{equation*}
    \end{enumerate}
    

    %- 7 -%
    \item 
    \textbf{Solución:}
    Sabemos que $\phi = 7.52 \times 10^{-19}J$ y que la $\lambda_{incidente} = 36 \times 10^{-9}m$
    $$ E_{inc} = \phi + E_k$$
    \begin{enumerate}
        \item 
        \begin{equation*}
            \begin{split}
                E_{km\acute{a}x} &= \frac{(6.626 \times 10^{-34}J.s)(3 \times 10^{8}m/s)}{36 \times 10^{-9}m} -  7.52 \times 10^{-19}J\\
                E_{km\acute{a}x} &= 4.77 \times 10^{-18}J
            \end{split}
        \end{equation*}
    \end{enumerate}


    %- 8 -%
    \item 
    \textbf{Solución:}
    Se sabe que el $e^{-}$ se encuentra en el cuarto nivel de energ\'ia y emite una energ\'ia $E = 4.16 \times 10^{-19}J$ 
    \begin{enumerate}
        \item 
        \begin{equation*}
            \begin{split}
                E &= 4.16 \times 10^{-19}J = 2.18 \times 10^{-18} \left(\frac{1}{n_1^{2}} - \frac{1}{4^{2}}\right)\\
                n_1 &= 1.98 
                n_1 \thickapprox 2
            \end{split}
        \end{equation*}
    
        \item
        \begin{equation*}
            \begin{split}
                n\lambda & = 2\pi r_n\\
                4\lambda & = 2\pi r_4\\
                \lambda & = \frac{2\pi r_4}{4}\\ 
                \lambda & = \frac{\pi r_4}{2} 
            \end{split}
        \end{equation*}

        \item
        \begin{equation*}
            \begin{split}
                r_n &= \frac{n^2}{z}a_0\\
                r_4 - r_2 &= 16a_0 - 4a_0\\
                r_4 - r_2 &= 12a_0\\
                \Delta r &= 12(0.529) = 6.348
            \end{split}
        \end{equation*}
    
    \end{enumerate}


    %- 9 -%
    \item 
    \textbf{Solución:}
    Sabemos que por la ecuaci\'on de De Broglie: $ \lambda = \frac{h}{p}$, entonces podemos inferir la siguiente formula:
    $$ \lambda = \frac{h}{\sqrt{2m_{e}e^{-}V_0}} $$
    \begin{enumerate}
        \item 
        \begin{equation*}
            \begin{split}
                \lambda &= \frac{6.626 \times 10^{-34}J.s}{\sqrt{2(9.1 \times 10^{-31}Kg)(1.6 \times 10^{-19}C)(400V)}}\\
                \lambda &= 6.14 \times 10{-11}m
            \end{split}
        \end{equation*}
    \end{enumerate}


    %- 10 -%
    \item 
    \textbf{Solución:}
    $$X^{1+}: 1s^{2}2s^{2}2p^{6}3s^{2}3p^{6}4s^{2}$$
    $${e_1}^{-} = \left(4,0,0,+\frac{1}{2}\right)$$
    $${e_2}^{-} = \left(4,0,0,-\frac{1}{2}\right)$$


    %- 11 -%
    \item 
    \textbf{Solución:}\\
    Rpta. B)


    %- 12 -%
    \item 
    \textbf{Solución:}\\
    Dato: $n + l = 4$
    $$ n = \{1, 2, 3, 4, 5,\ldots\}$$
    $$ l = \{0, 1, 2, 3, 4,\ldots\}$$
    Valores admisibles para $n$ y $l$ son: $\{(3,1);(4,0)\} ; n>l$\\
    n\'umero de $e^{-}$ del mismo espin = 4


    %- 13 -%
    \item 
    \textbf{Solución:}\\
    Rpta. A) y C)


    %- 14 -%
    \item 
    \textbf{Solución:}\\
    Energ\'ia de Ionizaci\'on:
    ${}_{12}Z<{}_5X<{}_9Y$\\
    Radio At\'omico:
    ${}_9Y<{}_5X<{}_{12}Z$


    %- 15 -%
    \item 
    \textbf{Solución:}\\
    $^{82}Pb: [Xe]_{54}6s^{2}4f^{14}5d^{10}6p^{2}$\\
    Rpta. A) $6s^{2}6p^{2}$


    %- 16 -%
    \item 
    \textbf{Solución:}\\
    Rpta. C) $^{52}Te$

    %- 17 -%
    \item 
    \textbf{Solución:}\\
    Sean los elementos:
    $${}_{13}Al: [Ne]3s^{2}3p^{1} (Paramagn\acute{e}tico)$$
    $${}_{20}Ca: [Ar]4s^{2} (Diamagn\acute{e}tico)$$
    $${}_{25}Mn: [Ar]4s^{2}3d^{5} (Paramagn\acute{e}tico)$$


    %- 18 -%
    \item 
    \textbf{Solución:}\\
    A partir de la ecuaci\'on qu\'imica:
    $$Na^{1+} + Cl^{1-} \longrightarrow NaCl$$
    $$EI = 496 KJ/mol$$
    $$AE = -348 KJ/mol$$
    $$\Delta E = 148 KJ/mol$$


    %- 19 -%
    \item 
    \textbf{Solución:}\\
    Rpta. B) Nitrogenoide


    %- 20 -%
    \item 
    \textbf{Solución:}\\
    Sean los elementos:
    \begin{enumerate}
        \item 
        \begin{equation*}
            \begin{split}
                Ag^{+}: [Kr]5s^{1}4d^{10} \to \left(5,0,0,+\frac{1}{2}\right)\\
                O^{2-}: [He]2s^{2}2p^{2} \to \left(2,1,0,+\frac{1}{2}\right)\\
                W^{5+}: [Xe]6s^{2}4f^{9} \to \left(6,0,0,-\frac{1}{2}\right)            
            \end{split}
        \end{equation*}
        
        \item
        \begin{equation*}
            \begin{split}
                Ag^{+}: [Kr]5s^{1}4d^{10} ; (Paramagn\acute{e}tico)\\
                O^{2-}: [He]2s^{2}2p^{2} ; (Paramagn\acute{e}tico)\\
                W^{5+}: [Xe]6s^{2}4f^{9} ; (Paramagn\acute{e}tico)
            \end{split}
        \end{equation*}
    \end{enumerate}



\end{enumerate}

Continuar\'a$\ldots $

\LaTeX


\end{document}